% Options for packages loaded elsewhere
% Options for packages loaded elsewhere
\PassOptionsToPackage{unicode}{hyperref}
\PassOptionsToPackage{hyphens}{url}
\PassOptionsToPackage{dvipsnames,svgnames,x11names}{xcolor}
%
\documentclass[
  letterpaper,
  DIV=11,
  numbers=noendperiod]{scrartcl}
\usepackage{xcolor}
\usepackage{amsmath,amssymb}
\setcounter{secnumdepth}{-\maxdimen} % remove section numbering
\usepackage{iftex}
\ifPDFTeX
  \usepackage[T1]{fontenc}
  \usepackage[utf8]{inputenc}
  \usepackage{textcomp} % provide euro and other symbols
\else % if luatex or xetex
  \usepackage{unicode-math} % this also loads fontspec
  \defaultfontfeatures{Scale=MatchLowercase}
  \defaultfontfeatures[\rmfamily]{Ligatures=TeX,Scale=1}
\fi
\usepackage{lmodern}
\ifPDFTeX\else
  % xetex/luatex font selection
\fi
% Use upquote if available, for straight quotes in verbatim environments
\IfFileExists{upquote.sty}{\usepackage{upquote}}{}
\IfFileExists{microtype.sty}{% use microtype if available
  \usepackage[]{microtype}
  \UseMicrotypeSet[protrusion]{basicmath} % disable protrusion for tt fonts
}{}
\makeatletter
\@ifundefined{KOMAClassName}{% if non-KOMA class
  \IfFileExists{parskip.sty}{%
    \usepackage{parskip}
  }{% else
    \setlength{\parindent}{0pt}
    \setlength{\parskip}{6pt plus 2pt minus 1pt}}
}{% if KOMA class
  \KOMAoptions{parskip=half}}
\makeatother
% Make \paragraph and \subparagraph free-standing
\makeatletter
\ifx\paragraph\undefined\else
  \let\oldparagraph\paragraph
  \renewcommand{\paragraph}{
    \@ifstar
      \xxxParagraphStar
      \xxxParagraphNoStar
  }
  \newcommand{\xxxParagraphStar}[1]{\oldparagraph*{#1}\mbox{}}
  \newcommand{\xxxParagraphNoStar}[1]{\oldparagraph{#1}\mbox{}}
\fi
\ifx\subparagraph\undefined\else
  \let\oldsubparagraph\subparagraph
  \renewcommand{\subparagraph}{
    \@ifstar
      \xxxSubParagraphStar
      \xxxSubParagraphNoStar
  }
  \newcommand{\xxxSubParagraphStar}[1]{\oldsubparagraph*{#1}\mbox{}}
  \newcommand{\xxxSubParagraphNoStar}[1]{\oldsubparagraph{#1}\mbox{}}
\fi
\makeatother


\usepackage{longtable,booktabs,array}
\usepackage{calc} % for calculating minipage widths
% Correct order of tables after \paragraph or \subparagraph
\usepackage{etoolbox}
\makeatletter
\patchcmd\longtable{\par}{\if@noskipsec\mbox{}\fi\par}{}{}
\makeatother
% Allow footnotes in longtable head/foot
\IfFileExists{footnotehyper.sty}{\usepackage{footnotehyper}}{\usepackage{footnote}}
\makesavenoteenv{longtable}
\usepackage{graphicx}
\makeatletter
\newsavebox\pandoc@box
\newcommand*\pandocbounded[1]{% scales image to fit in text height/width
  \sbox\pandoc@box{#1}%
  \Gscale@div\@tempa{\textheight}{\dimexpr\ht\pandoc@box+\dp\pandoc@box\relax}%
  \Gscale@div\@tempb{\linewidth}{\wd\pandoc@box}%
  \ifdim\@tempb\p@<\@tempa\p@\let\@tempa\@tempb\fi% select the smaller of both
  \ifdim\@tempa\p@<\p@\scalebox{\@tempa}{\usebox\pandoc@box}%
  \else\usebox{\pandoc@box}%
  \fi%
}
% Set default figure placement to htbp
\def\fps@figure{htbp}
\makeatother





\setlength{\emergencystretch}{3em} % prevent overfull lines

\providecommand{\tightlist}{%
  \setlength{\itemsep}{0pt}\setlength{\parskip}{0pt}}



 


\usepackage{graphicx, eso-pic, geometry}

% 1. Define the page geometry (A4)
\geometry{
    a4paper,
    top=35mm,    % Leaves room for the banner + some white space
    left=25mm,
    right=25mm,
    bottom=30mm
}

% 2. Absolute Positioning Logic
% This 'pins' the image to the Upper Left Corner of the physical paper
\AddToShipoutPictureBG{
    \AtPageUpperLeft{
        \raisebox{-\height}{
            \put(15mm, -30mm){
                \includegraphics[width=0.8\paperwidth]{hi-audkenni_29-idnveltold}
            }
        }
    }
}


% 3. Clean up the rest of the page (not page 1)
\usepackage{fancyhdr}
\pagestyle{fancy}
\fancyhf{} % Clears default headers/footers
\lfoot{\leftfootertext} % USES THE COMMAND FROM YAML
\cfoot{\thepage}
\rfoot{\rightfootertext}
\renewcommand{\headrulewidth}{0pt} % Removes the header line

% 4. THE FIX FOR PAGE 1: Redefine 'plain' style
% LaTeX automatically applies 'plain' to the title page. 
% We force 'plain' to look exactly like our 'fancy' style.
\fancypagestyle{plain}{
    \fancyhf{} % Clears default headers/footers
    \lfoot{\leftfootertext} % USES THE COMMAND FROM YAML
    \cfoot{\thepage}
    \rfoot{\rightfootertext}
    \renewcommand{\headrulewidth}{0pt} % Removes the header line
}
\KOMAoption{captions}{tableheading}
\newcommand{\leftfootertext}{VÉL407G vor 2026}
\newcommand{\rightfootertext}{Skilaverkefni 1}
\makeatletter
\@ifpackageloaded{caption}{}{\usepackage{caption}}
\AtBeginDocument{%
\ifdefined\contentsname
  \renewcommand*\contentsname{Table of contents}
\else
  \newcommand\contentsname{Table of contents}
\fi
\ifdefined\listfigurename
  \renewcommand*\listfigurename{List of Figures}
\else
  \newcommand\listfigurename{List of Figures}
\fi
\ifdefined\listtablename
  \renewcommand*\listtablename{List of Tables}
\else
  \newcommand\listtablename{List of Tables}
\fi
\ifdefined\figurename
  \renewcommand*\figurename{Figure}
\else
  \newcommand\figurename{Figure}
\fi
\ifdefined\tablename
  \renewcommand*\tablename{Table}
\else
  \newcommand\tablename{Table}
\fi
}
\@ifpackageloaded{float}{}{\usepackage{float}}
\floatstyle{ruled}
\@ifundefined{c@chapter}{\newfloat{codelisting}{h}{lop}}{\newfloat{codelisting}{h}{lop}[chapter]}
\floatname{codelisting}{Listing}
\newcommand*\listoflistings{\listof{codelisting}{List of Listings}}
\makeatother
\makeatletter
\makeatother
\makeatletter
\@ifpackageloaded{caption}{}{\usepackage{caption}}
\@ifpackageloaded{subcaption}{}{\usepackage{subcaption}}
\makeatother
\usepackage{bookmark}
\IfFileExists{xurl.sty}{\usepackage{xurl}}{} % add URL line breaks if available
\urlstyle{same}
\hypersetup{
  pdftitle={Skilaverkefni 1},
  colorlinks=true,
  linkcolor={blue},
  filecolor={Maroon},
  citecolor={Blue},
  urlcolor={Blue},
  pdfcreator={LaTeX via pandoc}}


\title{Skilaverkefni 1}
\usepackage{etoolbox}
\makeatletter
\providecommand{\subtitle}[1]{% add subtitle to \maketitle
  \apptocmd{\@title}{\par {\large #1 \par}}{}{}
}
\makeatother
\subtitle{VÉL407G Hönnun vélbúnaðar - vor 2026}
\author{}
\date{}
\begin{document}
\maketitle

\vspace{-20mm}
\begin{center}
  Aðferð og rökstudd niðurstaða gildir 70\%.\\
  Frágangur og framsetning lausnar gildir 30\%.
\end{center}


\section{Dæmi 1}\label{duxe6mi-1}

Gegnheill öxull með þvermál \(d\) verður fyrir bæði jafndreifðri
normalspennu \(\sigma_n\) vegna áskrafts \(P\) sem og beygjuspennu
\(\sigma_b\) vegna vægis \(M\). Hæsta spenna á ytra yfirborði öxuls er
\(\sigma_{max}=\sigma_n+\sigma_b\).

Áskraftur \(P\) hefur meðaltal \(\mu_P=86.6 \text{ kN}\) og staðalfrávik
\(\hat{\sigma}_P = 8.08 \text{ kN}\). Beygjuvægið hefur meðaltal
\(\mu_M=1610 \text{ kNm}\) og staðalfrávik
\(\hat{\sigma}_M=93.9 \text{ kNm}\).

Öxull er úr stáli sem hefur meðalflotstyrk
\(\bar{S}_y = 553 \text{ MPa}\) og staðalfrávik
\(\hat{\sigma}_{S_y}=42.7 \text{ MPa}\).

Gerið ráð fyrir að allar stærðir séu óháðar og normaldreifðar.

Ákvarðið hönnunarstuðul og þvermál öxuls m.v. 99\% áreiðanleika (öryggi)
gegn flotbjögun.

\section{Dæmi 2}\label{duxe6mi-2}

Eftirfarandi biti \(OABC\) ber álag í \(xy\) og \(zx\) plönum.

\begin{enumerate}
\def\labelenumi{(\alph{enumi})}
\tightlist
\item
  Setjið upp Macauley föll fyrir álag á bitann í \(xy\) og \(yz\)
  plönum.
\item
  Finnið undirstöðukrafta í \(y\) og \(z\) stefnur í punktum \(O\) og
  \(C\).
\item
  Teiknið sker- og vægisferla fyrir bæði \(xy\) og \(zx\) plön. Berið
  kennsl á krístíska punkta fyrir skerkraft og beygjuvægi.
\item
  Finnið hæstu beygjuspennur (tog og þrýsti) m.v. gefið þversnið og
  merkið viðeigandi staðsetningar í þvernsiði.
\end{enumerate}

\begin{center}
\includegraphics[width=0.8\linewidth,height=\textheight,keepaspectratio]{fig/v1_d2.png}
\end{center}

\newpage

\section{Dæmi 3}\label{duxe6mi-3}

Öxull \(ABD\) hefur gegnheilt hringlaga þversnið með þvermál
\(d=12 \text{ mm}\). Stangir \(BE\) og \(CD\) eru soðnar á öxul.

\begin{enumerate}
\def\labelenumi{(\alph{enumi})}
\tightlist
\item
  Ákvarðið undirstöðukrafta og vægi í punkti A.
\item
  Teiknið skerkrafts og vægisferla.
\item
  Ákvarðið staðsetningu á krítískum spennum í þversniði öxuls í punkti
  A.
\item
  Ákvarðið höfuðspennur ásamt samsvarandi hornum höfuðplana og hæstu
  skerspennur í krítískum punkti úr (c) lið.
\end{enumerate}

\begin{center}
\includegraphics[width=0.8\linewidth,height=\textheight,keepaspectratio]{fig/v1_d3.png}
\end{center}

\newpage

\section{Dæmi 4}\label{duxe6mi-4}

Biti \(OAB\) er innspenntur í punkti \(O\) og myndaður með því að festa
saman tvö rétthyrnd horn sem bæði hafa hliðarlengdir
\(w=125 \text{ mm}\) og þykkt \(t=10.0 \text{ mm}\). Á bitann verkar
jafndreift álag \(q=1 \text{ kN/m}\) yfir \(OAB\) og stakur kraftur
\(P_A=2.5 \text{ kN}\) í punkti A.

\begin{enumerate}
\def\labelenumi{(\alph{enumi})}
\tightlist
\item
  Ákvarðið lóðrétta færslu \(y(x)\) í punkti \(x=B\) með
  samlagningaraðferð (e. superposition).
\item
  Endurtakið lið (a) en notið Macauley sérstöðuföll.
\item
  Berið saman niðurstöður í (a) og (b).
\item
  Teiknið \(y(x)\) og \(\theta(x)\) þar sem \(\theta\) er halli (e.
  slope) bitans.
\end{enumerate}

\begin{center}
\includegraphics[width=0.8\linewidth,height=\textheight,keepaspectratio]{fig/v1_d4.png}
\end{center}




\end{document}
